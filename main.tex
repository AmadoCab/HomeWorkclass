\documentclass[theorems,spanish,code]{HomeWork}
\usepackage[utf8]{inputenc}
\usepackage[spanish]{babel}
\usepackage{tikz}
\usetikzlibrary{calc}

\usepackage{kantlipsum}
\setlength{\parindent}{0pt}

\definecolor{celestoso}{HTML}{6D90F5}
\definecolor{sky}{HTML}{1E3984}
\definecolor{silvblue}{HTML}{6F6F6F}
\definecolor{grisito}{HTML}{D6D6D6}
\definecolor{bblue}{HTML}{A7B3D8}

%\titlesty{classic}

\institute{Escuela de Ciencias Físicas y Matemáticas}
\title{Tarea \#1}
\author{Alonso Quijano el Bueno}
\id{202102022}
\email{nombredeusuario@gmail.com}
\course{Análisis de Algoritmos}
\professor{Miguel de Cervantes}
\date{\today}

\begin{document}

\maketitle

\part{Partes}
\section{Secciones}
\subsection{Subsecciones}
\subsubsection{Subsubsecciones}
\paragraph{\kant[1]}

\section{Texto}
\kant[2-3]

\section{Listas}
\begin{minipage}{0.5\textwidth}
\subsection{Numeradas}
\begin{enumerate}
    \item primer elemento
    \staritem segundo elemento
    \item tercer elemento
    \begin{enumerate}
        \staritem primer nest
        \begin{enumerate}
            \item segundo nest
            \begin{enumerate}
                \staritem tercer nest
            \end{enumerate}
        \end{enumerate}
    \end{enumerate}
    \item cuarto elemento
\end{enumerate}

\subsection{No numeradas}
\begin{itemize}
    \item primer elemento
    \item segundo elemento
    \item tercer elemento
    \begin{itemize}
        \item primer nest
        \begin{itemize}
            \item segundo nest
            \begin{itemize}
                \item tercer nest
            \end{itemize}
        \end{itemize}
    \end{itemize}
    \item cuarto elemento
\end{itemize}
\end{minipage}
\begin{minipage}{0.5\textwidth}
\subsection{Descriptiva}
\begin{description}
    \item[primer] elemento
    \item[segundo] elemento
    \item[tercer] elemento
    \begin{description}
        \item[primer] nest
        \begin{description}
            \item[segundo] nest
            \begin{description}
                \item[tercer] nest
            \end{description}
        \end{description}
    \end{description}
    \item[cuarto] elemento
\end{description}
\end{minipage}

\section{Matemática}
\kant[4]
\begin{theorem}[Subtítulo]
For any summation method $L$, its Abelian theorem is the result that if $c=(c_n)$ is a convergent sequence, with limit $C$, then $L(c)=C$. An example is given by the Cesàro method, in which $L$ is defined as the limit of the arithmetic means of the first $N$ terms of $c$, as $N$ tends to infinity. One can prove that if $c$ does converge to $C$, then so does the sequence $(d_N)$ where
\end{theorem}
\kant[5]
\begin{theorem*}[Teorema personalizado]
For any summation method $L$, its Abelian theorem is the result that if $c=(c_n)$ is a convergent sequence, with limit $C$, then $L(c)=C$. An example is given by the Cesàro method, in which $L$ is defined as the limit of the arithmetic means of the first $N$ terms of $c$, as $N$ tends to infinity. One can prove that if $c$ does converge to $C$, then so does the sequence $(d_N)$ where
\end{theorem*}
\kant[6]
\begin{proof}
\kant[7]
\end{proof}

\newpage

\section{Boxed and color boxed}
\begin{minipage}{0.5\textwidth}
\subsection{Boxed}
\[\boxed{ \frac{-b\pm\sqrt{b^2-4ac}}{2a} }\]
\subsection{ABoxed}
\begin{equation*}
\begin{split}
    x^2 - 4 &= 0\\
    x &= 2
\end{split}
\end{equation*}
\end{minipage}
\begin{minipage}{0.5\textwidth}
\subsection{CBoxed}
\[\Cboxed{ \frac{-b\pm\sqrt{b^2-4ac}}{2a} }\]
\subsection{CABoxed}
\begin{equation*}
\begin{split}
    x^2 - 4 &= 0\\
    \CAboxed{x &= 2}
\end{split}
\end{equation*}
\end{minipage}

\kant[8]
\begin{nameEq}{Texto de label}
\frac{1}{a}x^2+bx+c=0
\end{nameEq}
\kant[9]
\labline{Separator}
\kant*[10]
\[ \text{Total derivative: }\td[3]{f}{x} \qquad \text{Differential: }\dif x \qquad \text{Total derivative: }\td*[3]{f}{x} \]

\[ \text{Partial derivative: }\pd[3]{f}{x,y,z} \qquad \text{Differential: }\pdif x \qquad \text{Partial derivative: }\pd*[3]{f}{x,y,z} \]
\kant*[11]
\begin{ans}
\kant*[12]
\end{ans}
\kant*[13]
\begin{prob}
\kant*[14]
\end{prob}
\kant*[15-16]
\begin{info}
\kant*[17]
\end{info}
\kant*[18]
\begin{warn}
\kant*[19]
\end{warn}
\kant*[20-22]
\begin{file}[04Fortran.f90]{fortran}
! File done following () tutorial.
program fortrantut
    implicit none
    real :: float_num = 1.111111111111111
    real :: float_num2 = 1.111111111111111
    double precision :: dbl_num = 1.111111111111111d+0
    double precision :: dbl_num2 = 1.111111111111111d+0
    real :: rand(1)
    integer :: low = 1, high = 10


    print "(a26)", "Making a lot of operations"
    print "(a8,i1)", "5 + 4 = ", (5+4)
    print "(a8,i1)", "5 - 4 = ", (5-4)
    print "(a8,i2)", "5 * 4 = ", (5*4)
    print "(a8,i1)", "5 / 4 = ", (5/4)
    print "(a8,i1)", "5 % 4 = ", mod(5,4)
    print "(a8,i3)", "5 ^ 4 = ", (5**4)

    print "(a33)", "Demonstrating how presition works"
    print "(f17.15)", float_num + float_num2
    print "(f17.15)", dbl_num + dbl_num2

    call random_number(rand)
    print "(i2)", low + floor((high+1-low)*rand)

end program fortrantut
\end{file}
con la salida.
\begin{commandline}[escapeinside=||]
$ cat 04Fortran.f90
$ ./a.out
Making a lot of operations
5 + 4 = 9
5 - 4 = 1
5 * 4 = 20
5 / 4 = 1
5 % 4 = 1
5 ^ 4 = 625
Demonstrating how presition works
2.222222328186035
2.222222222222222
 9
\end{commandline}
\kant*[23]

\newpage

{\color{green!50!black}\vrule width 3pt}
\vbox{%
\begin{adjustwidth}{2pt}{7pt}
    \kant*[14]\\
    
    – Amado C.
\end{adjustwidth}}

\end{document}
