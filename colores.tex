\documentclass[theorems,spanish]{Marianita}
\usepackage[utf8]{inputenc}
\usepackage[spanish]{babel}
\usepackage{tikz}
\usetikzlibrary{calc}

\usepackage{kantlipsum}
\setlength{\parindent}{0pt}

% \definecolor{first}{HTML}{1E3984}
% \definecolor{second}{HTML}{314E9F}
% \definecolor{third}{HTML}{4B64B3}
% \definecolor{fourth}{HTML}{6277CB}
% \definecolor{fifth}{HTML}{556AA6}
% \definecolor{sixth}{HTML}{5C5C5C}
% \definecolor{seventh}{HTML}{6E6E6E}
% \definecolor{eighth}{HTML}{B4B4B4}
% \definecolor{nineth}{HTML}{6D90F5}
% \definecolor{tenth}{HTML}{B7C8F9}
% \definecolor{eleventh}{HTML}{A7B3D8}
% \definecolor{twelfth}{HTML}{6C74D1}


\definecolor{first}{HTML}{455B1C}
\definecolor{second}{HTML}{597820}
\definecolor{third}{HTML}{658924}
\definecolor{fourth}{HTML}{6F9B1E}
\definecolor{fifth}{HTML}{3D4C21}
\definecolor{sixth}{HTML}{A88347}
\definecolor{seventh}{HTML}{B39657}
\definecolor{eighth}{HTML}{BFAA67}
\definecolor{nineth}{HTML}{A6C044}
\definecolor{tenth}{HTML}{97B84B}
\definecolor{eleventh}{HTML}{99A362}
\definecolor{twelfth}{HTML}{653912}

%\titlesty{classic}

\institute{Escuela de Ciencias Físicas y Matemáticas}
\title{Tarea \#1}
\author{Alonso Quijano el Bueno}
\id{202102022}
\email{nombredeusuario@gmail.com}
\course{Análisis de Algoritmos}
\professor{Miguel de Cervantes}
\date{\today}

\begin{document}

\maketitle

{\color{seventh}\part{\color{first}Título de la primera parte}}
{\color{second}\section{Secciones}}
{\color{third}\subsection{Subsecciones}}
{\color{fourth}\subsubsection{Subsubsecciones}}
\paragraph{\kant[1]}.

{\color{second}\section{Texto}}
\kant[2-3]

{\color{second}\section{Listas}}
\begin{minipage}{0.5\textwidth}
{\color{third}\subsection{Numeradas}}
\begin{enumerate}
    \item primer elemento
    \item segundo elemento
    \item tercer elemento
    \begin{enumerate}
        \item primer nest
        \begin{enumerate}
            \item segundo nest
            \begin{enumerate}
                \item tercer nest
            \end{enumerate}
        \end{enumerate}
    \end{enumerate}
    \item cuarto elemento
\end{enumerate}

{\color{third}\subsection{No numeradas}}
\begin{itemize}
    \item primer elemento
    \item segundo elemento
    \item tercer elemento
    \begin{itemize}
        \item primer nest
        \begin{itemize}
            \item segundo nest
            \begin{itemize}
                \item tercer nest
            \end{itemize}
        \end{itemize}
    \end{itemize}
    \item cuarto elemento
\end{itemize}
\end{minipage}
\begin{minipage}{0.5\textwidth}
{\color{third}\subsection{Descriptiva}}
\begin{description}
    \item[primer] elemento
    \item[segundo] elemento
    \item[tercer] elemento
    \begin{description}
        \item[primer] nest
        \begin{description}
            \item[segundo] nest
            \begin{description}
                \item[tercer] nest
            \end{description}
        \end{description}
    \end{description}
    \item[cuarto] elemento
\end{description}
\end{minipage}

{\color{second}\section{Matemática}}
\kant[4]
\begin{theorem}[Subtítulo]
For any summation method $L$, its Abelian theorem is the result that if $c=(c_n)$ is a convergent sequence, with limit $C$, then $L(c)=C$. An example is given by the Cesàro method, in which $L$ is defined as the limit of the arithmetic means of the first $N$ terms of $c$, as $N$ tends to infinity. One can prove that if $c$ does converge to $C$, then so does the sequence $(d_N)$ where
\end{theorem}
\kant[5]
\begin{theorem*}[Teorema personalizado]
For any summation method $L$, its Abelian theorem is the result that if $c=(c_n)$ is a convergent sequence, with limit $C$, then $L(c)=C$. An example is given by the Cesàro method, in which $L$ is defined as the limit of the arithmetic means of the first $N$ terms of $c$, as $N$ tends to infinity. One can prove that if $c$ does converge to $C$, then so does the sequence $(d_N)$ where
\end{theorem*}
\kant[6]
\begin{proof}
\kant[7]
\end{proof}

\newpage

{\color{second}\section{Boxed and color boxed}}
\begin{minipage}{0.5\textwidth}
{\color{third}\subsection{Boxed}}
\[\boxed{ \frac{-b\pm\sqrt{b^2-4ac}}{2a} }\]
{\color{third}\subsection{ABoxed}}
\begin{equation*}
\begin{split}
    x^2 - 4 &= 0\\
    x &= 2
\end{split}
\end{equation*}
\end{minipage}
\begin{minipage}{0.5\textwidth}
{\color{third}\subsection{CBoxed}}
\[\Cboxed{ \frac{-b\pm\sqrt{b^2-4ac}}{2a} }\]
{\color{third}\subsection{CABoxed}}
\begin{equation*}
\begin{split}
    x^2 - 4 &= 0\\
    \CAboxed{x &= 2}
\end{split}
\end{equation*}
\end{minipage}

\kant[8]
\begin{nameEq}{Texto de label}
\frac{1}{a}x^2+bx+c=0
\end{nameEq}
\kant[9]
\labline{Separator}
\kant*[10]
\[ \text{Total derivative: }\td[3]{f}{x} \qquad \text{Differential: }\dif x \qquad \text{Total derivative: }\td*[3]{f}{x} \]

\[ \text{Partial derivative: }\pd[3]{f}{x,y,z} \qquad \text{Differential: }\pdif x \qquad \text{Partial derivative: }\pd*[3]{f}{x,y,z} \]
\kant*[11]
\begin{ans}
\kant*[12]
\end{ans}
\kant*[13]\\

{\color{tenth}\vrule width 3pt}
\vbox{%
\begin{adjustwidth}{2pt}{7pt}
    \kant*[14]\\
    
    – Amado C.
\end{adjustwidth}}

\end{document}
